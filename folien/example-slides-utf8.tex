% -*- coding: utf-8 -*-
\documentclass[t]{beamer} % Add option ``handout'' to produce handouts
\usepackage[english, ngerman]{babel}
\usepackage[utf8]{inputenc}
\usepackage[T1]{fontenc}
\usepackage{pxfonts}

\usetheme[german, informal]{UZH} % Available options: german (default), english,
                                 % informal (default), formal

\mode<presentation>
{
  \setbeamercovered{transparent}

  % For gray section slides
  \AtBeginSection[]{%
    \setbeamercolor{background canvas}{bg=UZHgray}
    \setbeamercolor{frametitle}{bg=UZHgray, fg=white}
    \begin{frame}<beamer>[plain]
      \frametitle{\insertsection}
    \end{frame}
    \setbeamercolor{background canvas}{bg=white}
    \setbeamercolor{frametitle}{bg=white, fg=UZHblue}
  }
}

\mode<handout>
{
  \usepackage{pgfpages}
  \pgfpagesuselayout{2 on 1}[a4paper, border shrink=5mm]
}

\newcommand{\first}[1]{\emph{#1}}
\newcommand{\q}[1]{\iflanguage{ngerman}{\flqq#1\frqq}{``#1''}}

%%%%%%%%%%%%%%%%%%%%%%%%%%%%%%%%%%%%%%%%%%%%%%%%%%%%%%%%%%%%%%%%%%%%%%%%%%%%%%%

\title[Präsentationen]{Präsentationen für Seminare}
\subtitle{Seminar \q{Making Word Processors Process Words} (HS~2009)}
\institute{Institut für Computerlinguistik}
\author{Michael Piotrowski\\
  \href{mailto:mxp@cl.uzh.ch}{\ttfamily mxp@cl.uzh.ch}}
\date{\today}

\begin{document}
% NOTE: Do not enclose \maketitle in a frame, or switching to the
%       normal templates will not work!
\maketitle

\begin{frame}
  \frametitle{Beamer}

  \begin{itemize}
  \item Beamer ist ein \LaTeX-Paket für Präsentationen
  \item Beamer hat eine irrsinnige Zahl von Optionen, meistens braucht
    man aber nicht viel mehr, als was in dieser Beispieldatei
    verwendet wird.
  \item Bei Bedarf siehe Handbuch:

    \url{http://ctan.org/get/macros/latex/contrib/beamer/doc/beameruserguide.pdf}
  \item Die UZH-Beamer-Vorlage (das sog. \first{Theme}) wurde von
    Michael Piotrowski entwickelt.
  \end{itemize}
\end{frame}

\begin{frame}
  \frametitle{Listen}

  Die folgenden Punkte werden nach und nach aufgedeckt:
  
  \begin{itemize}[<+->]
  \item Foo
  \item Bar
  \item Baz
  \end{itemize}

  Diese dagegen erscheinen aufs mal:

  \begin{itemize}[<*>]
  \item Foo
  \item Bar
  \item Baz
  \end{itemize}
\end{frame}

\begin{frame}
  \frametitle{Hervorhebungen}

  \begin{itemize}[<+->]
  \item This is \alert<.>{important}.
  \item We want to \alert<.>{highlight} this and \alert<.>{this}.
  \item What is the \alert<.>{matrix}?
  \end{itemize}
\end{frame}

\section{Abschnittstitel}

\begin{frame}
  \frametitle{Grafiken}

  Beispiel für eine Grafik:

  \includegraphics[width=\textwidth]{graphics/dilbert-unix}

  \begin{itemize}
  \item Breite und Höhe muss man je nach Format u.\,U. manuell
    anpassen.
  \item Abbildungen auf Folien sollten grundsätzlich so groß wie
    möglich sein.
  \end{itemize}
\end{frame}

\begin{frame}[fragile]
  \frametitle{Codebeispiele}

  \begin{itemize}
  \item Wenn man auf einer Folie \verb|\verb|, die
    \verb|verbatim|-Umgebung oder \verb|lstlisting| usw. aus dem
    \texttt{listings}-Paket benutzt, \alert{muss} man die Option
    \verb|fragile| angeben:

\begin{verbatim}
\begin{frame}[fragile]
\end{verbatim}
  \end{itemize}
\end{frame}

\begin{frame}[fragile]
  \frametitle{Handouts}

  \begin{itemize}
  \item Diese Vorlage ist für Handouts (zwei Folien pro Seite)
    vorbereitet.
  \item Zur Erstellung von Handouts:

    \verb|\documentclass[t]{beamer}|

    in

    \verb|\documentclass[t, handout]{beamer}|

    ändern.
  \end{itemize}
\end{frame}

\begin{frame}
  \frametitle{Weitere Informationen}

  \begin{itemize}
  \item
    \href{http://ctan.org/get/macros/latex/contrib/beamer/doc/beameruserguide.pdf}{Beamer-Handbuch}
    (\url{http://www.ctan.org/tex-archive/macros/latex/contrib/beamer/doc/beameruserguide.pdf})
  \item Herbert Voß (2009). Präsentationen mit LaTeX. Berlin:
    Lehmanns.
  \item Web
  \end{itemize}
\end{frame}
\end{document}

%%% Local variables:
%%% TeX-PDF-mode: t
%%% End:
