% -*- coding: utf-8 -*-
\documentclass[t]{beamer} % Add option ``handout'' to produce handouts
\usepackage[english, ngerman]{babel}
\usepackage[utf8]{inputenc}
\usepackage[T1]{fontenc}
\usepackage{pxfonts}

\usetheme[german, informal]{UZH} % Available options: german (default), english,
                                 % informal (default), formal

\mode<presentation>
{
  \setbeamercovered{transparent}

  % For gray section slides
  \AtBeginSection[]{%
    \setbeamercolor{background canvas}{bg=UZHgray}
    \setbeamercolor{frametitle}{bg=UZHgray, fg=white}
    \begin{frame}<beamer>[plain]
      \frametitle{\insertsection}
    \end{frame}
    \setbeamercolor{background canvas}{bg=white}
    \setbeamercolor{frametitle}{bg=white, fg=UZHblue}
  }
}

\mode<handout>
{
  \usepackage{pgfpages}
  \pgfpagesuselayout{2 on 1}[a4paper, border shrink=5mm]
}

\newcommand{\first}[1]{\emph{#1}}
\newcommand{\q}[1]{\iflanguage{ngerman}{\flqq#1\frqq}{``#1''}}

%%%%%%%%%%%%%%%%%%%%%%%%%%%%%%%%%%%%%%%%%%%%%%%%%%%%%%%%%%%%%%%%%%%%%%%%%%%%%%%

\title[Sprachvarianten]{Sprachvarianten des Deutschen im Zeitraum 1600 bis heute}
\subtitle{Vorlesung: Sprachtechnologie für historische Dokumente: Konzepte und Anwendungen (HS~2011)}
\institute[Institut für Computerlinguistik]{Institut für Computerlinguistik\\
Dozenten: Dr. Cerstin Mahlow, Dr.-Ing. Michael Piotrowski}
\author[Hafner, Marques~Madeira, Baumgartner]{Simon Hafner, Hernani Marques~Madeira, Reto Baumgartner}
\date{\today}



\begin{document}
% NOTE: Do not enclose \maketitle in a frame, or switching to the
%       normal templates will not work!
\maketitle

\section*{Problemstellung}

\begin{frame}
  \frametitle{Problemstellung}
  Erkennung der Entstehungszeit \\
  historischer Texte nach 1600
  \vspace*{1ex}
  
  Lösungsansatz mit Hilfe eines n"=Gramm"=Sprachmodells
\end{frame}

\section*{Trainingskorpus}

\begin{frame}
  \frametitle{Daten für das Trainingskorpus}
  Digitale Bibliothek
  \vspace*{1ex}
  
  Bereich Literatur
  \vspace*{1ex}
  
  Nach den Regeln der Text Encoding Initiative (TEI)
  
  \url{http://www.tei-c.org/index.xml}
\end{frame}

\begin{frame}
  \frametitle{TEI-Format}
  Mehrere Werke in einem XML"=File
  \vspace*{1ex}
  
  Einzelne Werke repräsentiert durch Knoten mit dem Tag «TEI»
  \begin{itemize}
  \item Literaturgenre:
    \begin{itemize}
    \item \texttt{TEI/teiHeader/profileDesc/textClass/keywords/term}    
    \item \texttt{prose}, \texttt{drama}, \texttt{verse}
    \end{itemize}
    \vspace*{1ex}
    \pause
    
  \item Erstellungsdatum: 
    \begin{itemize}
    \item \texttt{TEI/teiHeader/profileDesc/creation/date}   
    \item \texttt{notBefore=''1680'' notAfter=''1983''}
    \end{itemize}
    \vspace*{1ex}
    \pause
    
  \item Text:\\
    \begin{itemize}
    \item \texttt{TEI/text}
    \end{itemize}
  \end{itemize}
\end{frame}

\begin{frame}
  \frametitle{Werkzeug zur Erstellung des Trainingskorpus}
  \texttt{korpusbastler.py}
  \vspace*{1ex}
  
  In Python codiert
  \vspace*{1ex}
  
  Einsetzbar ab Python 3.2
  
  Gründe dazu:
  \begin{itemize}
  \item Eher neue Funktion \texttt{xml.etree.cElementTree.itertext()}
  \item Einfachere Arbeit mit Encodings ab Python 3.x
  \end{itemize}
\end{frame}

\begin{frame}
  \frametitle{Werkzeug zur Erstellung des Trainingskorpus}
  Für jedes enthaltene Werk:
  
  \begin{itemize}
  \item Literaturgenre:
    \begin{itemize}
    \item Weiterverarbeitung der Genres \item \texttt{prose}, \texttt{drama} 
    \item Keine Textextraktion bei Genres wie \texttt{verse}
    \end{itemize}
    \vspace*{1ex}
    \pause

  \item Erstellungsdatum:
    \begin{itemize}
    \item \texttt{notBefore, notAfter}: Lebensdaten des Autors
    \item Mögliches Erstellungsjahr: Mitte zwischen den Jahren 
    \item Einteilung in Korpora nach halben Jahrhunderten mithilfe des Erstellungsjahres
    \end{itemize}
    \vspace*{1ex}
    \pause
    
  \item Text:
    \begin{itemize}
    \item Extraktion des Textes auf allen Tiefen
    \item Mit \texttt{xml.etree.cElementTree.itertext()}
    \item Schreiben in entsprechende Korpusdateien
    \end{itemize}
  \end{itemize}
\end{frame}

\begin{frame}
  \frametitle{Trainingskorpora für die Sprachmodelle}
  \begin{tabular}{ll}
  Sprachstufe & Anzahl Wörter im Korpus
  \vspace*{1ex}\\
  1600--1650 & \\
  1650--1700 & \\
  1700--1750 & \\
  1750--1800 & \\
  1800--1850 & \\
  1850--1900 & \\
  1900--     & 
  \end{tabular}
\end{frame}

\section*{Testkorpus}

\begin{frame}
  \frametitle{Testkorpora}
  \begin{itemize}
  \item Aus \url{http://de.wikisource.org/}\pause
  \vspace*{1ex}
  \item 100 Sätze pro Sprachstufe\pause
  \item d.\,h. 20 Sätze pro Jahrzehnt\pause
  \vspace*{1ex}
  \item Genres entsprechen dem Trainingskorpus
  \end{itemize}  
\end{frame}


\end{document}

%%% Local variables:
%%% TeX-PDF-mode: t
%%% End:
